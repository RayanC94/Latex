\section{Conclusion}
\label{sec:conclusion}

\subsection{Synthèse des Contributions}

Ce projet avait pour objectif la reproduction de l'article scientifique de Zheng \& Lu (2017) sur l'estimation de la vigilance d'un conducteur à partir de signaux EEG et EOG. Nous avons mené à bien cette tâche en développant une implémentation complète et rigoureuse des trois modèles proposés : SVR, CCRF et CCNF.

\textbf{Résultats principaux :}

\begin{enumerate}
    \item \textbf{Reproduction réussie :}
    \begin{itemize}
        \item COR de 0.804 pour CCNF, soit \textbf{94.5\% de la performance de l'article} (0.85)
        \item Classement des modèles préservé : CCNF > SVR > CCRF
        \item Validation robuste sur 23 sujets avec 5-fold cross-validation
    \end{itemize}
    
    \item \textbf{Implémentation open-source :}
    \begin{itemize}
        \item Code modulaire et réutilisable en Python
        \item Documentation complète et visualisations professionnelles
        \item Architecture extensible facilitant les améliorations futures
    \end{itemize}
    
    \item \textbf{Analyse approfondie :}
    \begin{itemize}
        \item 6 visualisations détaillées des performances
        \item Identification des sujets faciles vs difficiles
        \item Étude de la variabilité inter-individuelle
    \end{itemize}
    
    \item \textbf{Discussion critique :}
    \begin{itemize}
        \item Analyse transparente des limitations (RMSE 2× supérieur)
        \item Identification des causes (sur-lissage, implémentation simplifiée)
        \item Proposition de pistes d'amélioration concrètes
    \end{itemize}
\end{enumerate}

\subsection{Atteinte des Objectifs}

\subsubsection{Objectifs Scientifiques}

\begin{table}[H]
\centering
\caption{Bilan des objectifs scientifiques}
\begin{tabular}{lcc}
\toprule
\textbf{Objectif} & \textbf{Statut} & \textbf{Commentaire} \\
\midrule
Reproduction des résultats & ✅ Atteint & COR à 94.5\% de l'article \\
Implémentation des 3 modèles & ✅ Atteint & SVR, CCRF, CCNF fonctionnels \\
Validation sur 23 sujets & ✅ Atteint & 115 expériences (23×5 folds) \\
Métriques identiques & ✅ Atteint & COR et RMSE calculés \\
Analyse comparative & ✅ Atteint & Graphiques et tableaux détaillés \\
\bottomrule
\end{tabular}
\end{table}

\subsubsection{Objectifs Techniques}

\begin{table}[H]
\centering
\caption{Bilan des objectifs techniques}
\begin{tabular}{lcc}
\toprule
\textbf{Objectif} & \textbf{Statut} & \textbf{Commentaire} \\
\midrule
Code modulaire & ✅ Atteint & Architecture src/ claire \\
Documentation & ✅ Atteint & Docstrings, README, rapport \\
Reproductibilité & ✅ Atteint & Seeds fixes, résultats stables \\
Visualisations & ✅ Atteint & 6 figures publication-ready \\
Gestion d'erreurs & ✅ Atteint & Robuste aux fichiers manquants \\
\bottomrule
\end{tabular}
\end{table}

\subsection{Limites Reconnues}

Nous reconnaissons les limitations suivantes de notre travail :

\begin{enumerate}
    \item \textbf{RMSE Élevé (0.192 vs 0.09)} :
    \begin{itemize}
        \item Implémentation simplifiée de CCRF/CCNF (lissage local vs Viterbi global)
        \item Impact : Précision absolue des prédictions à améliorer
        \item Pas critique pour la détection de tendances (COR reste élevé)
    \end{itemize}
    
    \item \textbf{Hyperparamètres Non Optimisés} :
    \begin{itemize}
        \item Pas de grid search exhaustive
        \item Paramètres fixés empiriquement
        \item Gain potentiel : 5-10\% d'amélioration
    \end{itemize}
    
    \item \textbf{Généralisation Cross-Subject Non Testée} :
    \begin{itemize}
        \item Validation par sujet (intra-subject)
        \item Généralisation à de nouveaux utilisateurs incertaine
        \item Nécessite validation leave-one-subject-out
    \end{itemize}
\end{enumerate}

\subsection{Perspectives Futures}

\subsubsection{Améliorations à Court Terme}

\begin{enumerate}
    \item \textbf{Optimisation du lissage temporel} :
    \begin{itemize}
        \item Réduction de la fenêtre (7 → 3 échantillons)
        \item Ajustement de $\beta$ (0.1 → 0.3-0.5)
        \item Objectif : RMSE réduit à 0.15-0.16
    \end{itemize}
    
    \item \textbf{Grid search d'hyperparamètres} :
    \begin{itemize}
        \item Exploration systématique de $C$, $\gamma$, $\epsilon$
        \item Optimisation par sujet si ressources disponibles
        \item Objectif : Gain de 5-10\% sur RMSE et COR
    \end{itemize}
\end{enumerate}

\subsubsection{Extensions à Moyen Terme}

\begin{enumerate}
    \item \textbf{Deep Learning avec LSTM} :
    \begin{itemize}
        \item Architecture récurrente pour temporalité
        \item Entraînement sur GPU (PyTorch/TensorFlow)
        \item Objectif : COR > 0.85, RMSE < 0.15
    \end{itemize}
    
    \item \textbf{Transformer avec Attention} :
    \begin{itemize}
        \item État de l'art en modélisation de séquences
        \item Interprétabilité via poids d'attention
        \item Objectif : Performance comparable à l'article (COR=0.85, RMSE=0.09)
    \end{itemize}
    
    \item \textbf{Validation Cross-Subject} :
    \begin{itemize}
        \item Leave-one-subject-out cross-validation
        \item Évaluation de la généralisation
        \item Domain adaptation si nécessaire
    \end{itemize}
\end{enumerate}

\subsubsection{Recherche à Long Terme}

\begin{enumerate}
    \item \textbf{Système Temps Réel} :
    \begin{itemize}
        \item Optimisation pour inférence rapide (<100ms)
        \item Interface de monitoring en direct
        \item Déploiement sur système embarqué
    \end{itemize}
    
    \item \textbf{Données Multi-Modales} :
    \begin{itemize}
        \item Intégration de données vidéo (visage)
        \item Fusion EEG + EOG + Vidéo + Conduite
        \item Robustesse accrue par redondance
    \end{itemize}
    
    \item \textbf{Étude Longitudinale} :
    \begin{itemize}
        \item Suivi sur plusieurs semaines/mois
        \item Adaptation au profil individuel
        \item Prédiction personnalisée
    \end{itemize}
\end{enumerate}

\subsection{Impact et Utilité}

\subsubsection{Contribution à la Recherche}

\begin{itemize}
    \item \textbf{Validation de la méthodologie} de Zheng \& Lu (2017)
    \item \textbf{Code open-source} réutilisable par la communauté
    \item \textbf{Base solide} pour extensions futures (LSTM, Transformer)
    \item \textbf{Documentation complète} facilitant la reproductibilité
\end{itemize}

\subsubsection{Applications Potentielles}

\begin{itemize}
    \item \textbf{Sécurité routière} : Prévention des accidents liés à la somnolence
    \item \textbf{Transport professionnel} : Monitoring des chauffeurs de poids lourds
    \item \textbf{Aviation} : Détection de la fatigue chez les pilotes
    \item \textbf{Industrie} : Surveillance des opérateurs de machines dangereuses
    \item \textbf{Santé} : Diagnostic des troubles du sommeil
\end{itemize}

\subsection{Réflexion Personnelle}

Ce projet m'a permis de :

\begin{enumerate}
    \item \textbf{Maîtriser le pipeline ML complet} : De la donnée brute à l'évaluation finale
    \item \textbf{Comprendre l'EEG/EOG} : Signaux physiologiques et leur traitement
    \item \textbf{Développer la rigueur scientifique} : Validation, métriques, reproductibilité
    \item \textbf{Identifier les limites} : Analyse critique et honnêteté intellectuelle
    \item \textbf{Proposer des améliorations} : Vision prospective et innovation
\end{enumerate}

\textbf{Leçon principale :} La reproduction d'article scientifique est un exercice formateur qui révèle les subtilités et les difficultés cachées derrière les résultats publiés. Notre COR de 0.80 (vs 0.85) démontre qu'une reproduction fidèle à 94\% est un excellent résultat, même si une partie des détails d'implémentation manque dans la publication originale.

\subsection{Mot de la Fin}

La détection automatique de la somnolence au volant représente un enjeu majeur de santé publique. Notre travail démontre qu'il est possible d'estimer la vigilance d'un conducteur avec une précision élevée (COR=0.80) en utilisant des signaux EEG et EOG, ouvrant la voie à des systèmes embarqués de prévention des accidents.

Les améliorations futures, notamment l'utilisation de réseaux LSTM ou Transformer, pourraient permettre d'atteindre voire de dépasser les performances de l'article original. Le code open-source que nous avons développé constitue une base solide pour ces extensions.

\begin{center}
\textit{``The best way to predict the future is to create it.''} \\
--- Alan Kay
\end{center}

\vspace{1cm}

\begin{center}
\rule{10cm}{0.4pt}
\end{center}

\vspace{0.5cm}

\begin{center}
\textbf{Fin du Rapport}
\end{center}